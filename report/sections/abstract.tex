\begin{abstract}
%Suppose that one needs to know the dimensions of a package, but does not have a measuring tool.
%What most people do have is a smartphone, which has the potential to be a measuring tool.
%If a reference object of known size is placed on top of the package, and the two objects are filmed together, the dimensions of the package can be calculated.

This report presents a method to measure the dimensions of cuboid packages automatically with a single uncalibrated view.
The method was designed to be used on a mobile device, and considers the consequent hardware and time constraints.

The method uses a planar reference object of known size, which is placed on top of the package.
The reference object and package are detected automatically in the image, and their positions are fed to a measuring algorithm.

The measuring algorithm uses the geometry of cuboids to avoid prior camera calibration.
By using the vanishing points which are induced by the edges of the package, the camera can be calibrated online from a single view.

With this method, a correct measurement was achieved in $92\%$ of the images when a subset of the tested camera poses were used. On the complete test set the success rate was $51\%$.
Online testing on a smartphone shows that the detection algorithm is not yet mature to be run in a real-time application, but could work well in a non-real time application.

%A version of the algorithm which used calibrated views were also implemented, to determine how much performance needed to be sacrificed to gain the convenience of not having to go through the calibration procedure.
%The success rates when testing the measuring algorithm in isolation was $81\%$ for uncalibrated views, and $93\%$ for calibrated views.

\end{abstract}