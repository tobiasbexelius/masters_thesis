\chapter{Related Work}
This chapter presents an overview of recent research in the most important fields related to this thesis: single view metrology, auto-calibration, and model-based object detection.

Single-view metrology allows measurements to be made from a single view, which is what this thesis aims to do.

Auto-calibration allows measurements to be made without having to calibrate the camera in advance (offline).
This will be done through vanishing point calibration.
Vanishing point calibration is often not considered to be auto-calibration as it requires special structure in the scene to work. 
However, since it can be performed automatically in this case where it is known that such structure will exist, it is described here nevertheless.

Model-based object detection was chosen because the well-defined shape of cuboids should make them easy to model.

\section{Single view metrology}
Single view metrology refers to the process of making measurements in images from a single view.
In general it is impossible to make 3D measurements from only a single view, since depth information is unknown.
If additional information about the scene is known, such as information about known distances and planes, certain measurements can be made.
For example, it is well established that planes can be rectified so that euclidean measurements can be made directly from the rectified image.
That is done by calculating a planar homography, using four image-world point correspondences of points in the plane.
Planar homographies are described further in section \ref{planar-homographies}.

It has been shown that other properties of the plane can be used instead of just point correspondences, such as known angles, length ratios in the plane, vanishing points, and vanishing lines \cite{liebowitz1998metric} \cite{criminisi2000single}.

Another potentially useful result in this area is presented in \cite{huang2004new}, where it is shown that measurements can also be made in planes perpendicular to the reference plane, provided that the camera is at least partially calibrated.

\section{Auto-calibration} \label{related_work:vanishing_point_calibration} 
As stated earlier, it is important that the calibration of the internal parameters of the camera is performed automatically, if the resulting application is to be convenient to use.
Preferably, it should be done without the user's knowledge, while the application is running.
This can be achieved through auto-calibration, which is the process of determining the internal camera parameters directly from a set of uncalibrated images.
This is done by imposing constraints on the internal parameters by automatically detecting corresponding points or other connections between multiple images. 
Depending on which constraints are used, and the number of images, the constraints form a linear or non-linear equation system \cite[p. 458-469]{hartley-zisserman}. 
As it will be shown in section \ref{camera-calibration}, it is possible to impose constraints on the internal parameters in the same way using only one view, by finding vanishing points in the scene.
This is done using the method presented in \cite[p. 195-226]{hartley-zisserman}, which uses three orthogonal vanishing points in a single view.

Several other methods for vanishing points calibration exist.
For example multi-view solutions such as in \cite{caprile1990using}, where two views are used along with a special cube which acts as a calibration object.
The cube has parallel lines drawn on its surfaces, which helps produce accurate vanishing points.
This procedure first determines the intrinsic parameters of each camera using the vanishing points formed by the cube, similarly to the method used here.
Then, the extrinsic parameters of the cameras are calculated in a second step.

There are also methods which aim to extract vanishing points automatically from any structured scene, without and special object in it \cite{cipolla1999camera} \cite{grammatikopoulos2007automatic}.
The only requirement is that there are vanishing points to find.
Luckily, vanishing points are common in man-made environments, both indoors and outdoors.

Some recent methods to perform auto-calibration, specifically designed for smartphones, use the orientation sensors of the phone in conjunction with images.
Most, if not all, modern  smartphones are equipped with gyros, accelerometers, and compasses, which can be used to determine its orientation.
In \cite{jia2014online} a method is presented to calibrate a smartphone camera and gyroscope simultaneously online, and to synchronise the camera and gyroscope data.
Synchronisation of camera and motion sensors enables computer vision and augmented reality applications to detect when the camera rotates.

\section{Model-based object detection} \label{related_work:object_detection}
The problem of detecting a package in an image is an object-class detection problem.
The goal is not to detect one specific item with known appearance, but rather a family of different object that belong to the same class.
In this case the class is cuboid packages (and white papers).

This is in general an extremely hard problem to solve for arbitrary objects, and suggested solutions to it are very complex.
One notable example is presented in \cite{yan20073d}. The idea is to construct a 3D model of an object in the class one wishes to detect.
This 3D model is created using a homographic framework from  multiple 2D views of the object.
A 2D image can then be tested against the 3D model by matching features between the image and the model. 
When testing against the PASCAL VOC dataset, a standard dataset for object classification and detection algorithms, the presented algorithm achieved an average precision (i.e. the rate at which selected items are relevant) of 0.144 and 0.182 for two classes of objects (horses and motorcycles).
That may not sound very impressive, but it should be noted that the PASCAL VOC challenge dataset is made to be difficult and the objects appear in a great variety of poses, scales, clutter, occlusion and otherwise poor conditions \cite{everingham2010pascal}.

More closely related to this thesis is the paper "Localizing 3D cuboids in single-view images", where the authors focus on the more delimited task of detecting rectangular cuboids in uncalibrated single-view images \cite{xiaolocalizing}.
Edge and corner information is used create a model which is trained using Support Vector Machines.
The method shows promising results and can handle different viewpoints, aspect ratio, scale, and occlusion.
Several others try to reason about man-made environments by detecting cuboid structures, for example: \cite{hedau2010thinking}, \cite{hedau2012recovering}, \cite{gupta2010estimating}, \cite{del2012bayesian}.

In this thesis, a simpler approach is used, because the detector does not have to deal with clutter or occlusion, because of the assumptions made in section \ref{introduction:delimitations}.
