Sometimes you want to know the measurements of a package and you might not have a measuring tool with you, but you always have your smartphone in your pocket! Imagine if you could have an app that would calculate the size of the package, no measuring tape or ruler needed. Just put a reference object like a credit card on top of the package, open up the app and film them together. The app then tells you the package’s dimensions. 
\section{Problem Statement}
The main problem that this project will investigate is:
\begin{itemize}
	\item Is it possible to measure a rectangular package by placing a reference object on top of it and filming the objects together with a cell phone camera, with adequate precision, robustness, and speed in order to make a mobile app with an acceptable degree of usability?
	\item The measuring process should not require user intervention such as marking the corners of the package or the reference object, they will be recognized automatically by the program.
\end{itemize}

If time allows this project will also investigate if it is possible to improve performance by using multiple images of the package from different perspectives.  Either by calculating a mean between several measurements or by using multiple images in the same calculation. 

Another possible extension is to try to use auto-calibration techniques, or simply to guess the intrinsic parameters of the camera instead of using regular checkerboard calibration, and how the accuracy of the different methods compare.
